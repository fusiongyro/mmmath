\starttext

\title{Survey of Modern Algebra}

Definition. Let $D$ be a set of elements $a, b, c, ...$ for which the sum $a + b$ and the product $ab$ of any two elements $a$ and $b$ (distinct or not) of $D$ are defined. Then $D$ is called an integral domain if the following postulates (i)-(ix) hold:

\startitemize[r,packed]
\item Closure. If $a$ and $b$ are in $D$, then the sum $a + b$ and the product $ab$ are in $D$;
\item Uniqueness. If $a = a'$ and $b = b'$ in $D$, then $a + b = a' + b'$ and $ab = a'b'$;
\item Commutative laws. For all $a$ and $b$ in $D$, $a + b = b + a$, $ab = ba$;
\item Associative laws. For all $a$, $b$ and $c$ in $D$, $a + (b + c) = (a + b) + c$, $a(bc) = (ab)c$;
\item Distributive law. For all $a$, $b$ and $c$ in $D$, $a(b+c) = ab + ac$.
\item Zero. $D$ contains an element 0 such that $a + 0 = a$, for all $a$ in $D$;
\item Unity. $D$ contains an element 1 such that $a\cdot 1 = a$ for all $a$ in $D$.
\item Additive inverse. For each $a$ in $D$, the equation $a + x = 0$ has a solution $x$ in $D$;
\item Cancellation law. If $c \ne 0$ and $ca = cb$, then $a = b$.
\stopitemize

Reflexive law: $a = a$,

Symmetric law: If $a = b$ then $b = a$,

Transitive law: If $a = b$ and $b = c$, then $a = c$, valid for all $a$, $b$, and $c$ in $D$.

\vskip 1em

Rule 1. $(a + b)c$ = $ac + bc$, for all $a$, $b$, $c$ in $D$.

Rule 2. For all $a$ in $D$, $0 + a = a$ and $1 \cdot a = a$.

Rule 3. If $z$ in $D$ has the property that $a + z = a$ for all $a$ in $D$, then $z = 0$.

Rule 4. For all $a, b, c$ in $D$, $a + b = a + c$ implies $b = c$.

Rule 5. For each $a$, $D$ contains one and only one solution $x$ of the equation $a + x = 0$.

Rule 6. For $a$ and $b$ in $D$, there is one and only one x in $D$ with a + x = b.

Rule 7. For all $a$ in $D$, $a \cdot 0 = 0 = 0 \cdot a$.

Rule 8. If $u$ in $D$ has the property that $au = a$ for all $a$ in $D$, then $u = 1$.

Rule 9. For all $a$ and $b$ in $D$, $(-a)(-b) = ab$.

\stoptext
